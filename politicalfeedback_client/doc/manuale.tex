\documentclass[10pt,a4paper]{book}
\usepackage[italian]{babel}
\usepackage[utf8]{inputenc}
\usepackage{amsmath}
\usepackage{amsfonts}
\usepackage{amssymb}
\usepackage{graphicx}
\usepackage{url}
\author{Carlo Zanatto}
\title{PoliticalFeedback}
\begin{document}
\part{PoliticalFeedback, solo uno strumento di lavoro}

\chapter{Introduzione}

\section{Scenario attuale}

Lo scenario attuale mostra come siano nati molti gruppi il cui scopo è portare avanti dei principi di base indipendentemente uno dall'altro. Vi sono inoltre delle macroaree, provinciali, regionali e nazionali che necessitano di una coordinazione dei gruppi sia a livello di strumenti di lavoro che di natura applicativa dei principi.
Per sopperire alla mancanza sia di uno strumento di coordinazione, sia di una struttura, sono nati dei gruppi di lavoro che hanno impostato un sistema che ha cercato nel tempo di creare delle anagrafiche e di coordinarle attraverso numerosi strumenti.
Gli strumenti attuali però sono diversificati da gruppo a gruppo e vedono alle volte per uno stesso argomento l'utilizzo di più canali di comunicazione ognuno con caratteristiche utili ma dispersive.
Si sono quindi create strutture piramidali in cui gli organizzatori, coinvolgendo spesso una minoranza non garante di tutti gli attivisti, hanno preso per tutti delle decisioni per sopperire a questa mancanza ritenendo di dover "arrivare al risultato" rinunciando però alla prerogativa fondamentale della democrazia diretta che vede il coinvolgimento di tutti per le decisioni di competenza.
Si sono inoltre creati dei sistemi di votazione per le elezioni dei candidati non rappresentativi di tutti, escludendo chi non si è trovato d'accordo con tale procedura.



\section{Motivazioni}
La necessità di avere uno strumento efficace per condividere il lavoro politico e avere un feedback istantaneo di tutti i membri di un gruppo di lavoro mi ha portato a pensare a questo strumento. Questo strumento nasce con lo scopo di eliminare la catastrofica presenza di un amministratore, condizione che presuppone la creazione di una gerarchia di controllo che viene definita in maniera automatica nel momento in cui viene istituito questo ruolo. Ma nasce anche perchè il nostro ruolo è quello di fare la volontà dei cittadini e quindi di farli partecipare alle nostre decisioni, anche se non attivisti.

\section{Democrazia diretta}
Di seguito una piccola introduzione alla democrazia diretta\footnote{\url{https://it.wikipedia.org/wiki/Democrazia_diretta}} che vuole mostrare come in realtà questi concetti non siano assolutamente applicati nelle nostre procedure interne e come non vi sa uno strumento già pronto che permetta di metterli in pratica in maniera efficace nemmeno per noi stessi, prima di proporli ai cittadini.
\subsection{Principi di base}
La democrazia diretta è quella forma di democrazia nella quale anche i cittadini possono, nel rispetto delle regole previste, esercitare il potere legislativo.

\subsection{Elementi necessari alla sua applicazione}
Gli strumenti mediante i quali i cittadini possono esercitare la democrazia diretta sono i seguenti:
\begin{itemize}
\item 
\textbf{Petizione}, è lo strumento più semplice di democrazia diretta. Una petizione impone all'organo al quale viene indirizzato di dare una semplice risposta;
\item
\textbf{Referendum abrogativo}, tramite il quale i cittadini possono abrogare un provvedimento legislativo votato dai rappresentanti. L'abrogazione avviene chiamando al voto i cittadini stessi, per esempio mediante la raccolta di firme. Si parla talora di "referendum facoltativo";
\item
\textbf{Referendum obbligatorio}, tramite il quale i cittadini possono essere chiamati al voto anche senza preventiva raccolta di firme. Dove esiste, il referendum obbligatorio scatta automaticamente (ed in modo non facoltativo) per alcune tipologie specifiche di leggi. Per esempio per leggi che concernono direttamente i legislatori, come per esempio le leggi elettorali. Oppure per leggi che comportino voci di spesa particolarmente elevati. Questo referendum viene talvolta anche chiamato "Referendum confermativo";
\item
\textbf{Legge di iniziativa popolare a voto parlamentare}, permette, raccogliendo il numero di firme necessario nei tempi e nelle modalità previste, chiamano alla discussione ed al voto i rappresentanti eletti. Questo strumento di democrazia diretta à detto a volte anche "Proposta popolare" o "Mozione di iniziativa popolare";
\item
\textbf{Legge di iniziativa popolare a voto popolare}. In questo caso i propositori della legge di iniziativa popolare chiamano al voto direttamente l'insieme degli elettori. In molti casi l'organo legislativo ha il diritto di fare una contro-proposta. Questo strumento è a volte anche detto "Referendum propositivo";
\item
\textbf{Revoca}. Questo strumento di democrazia diretta consente ai cittadini, seguendo le specifiche procedure, di revocare un rappresentante eletto prima dello scadere del suo mandato oppure anche, eventualmente, tutto l'organo rappresentantivo.
\item
\textbf{Un discorso a parte merita il "Plebiscito"}, dove è il governante che chiama al voto i cittadini. Nella citazione di Auer (precedente) questo strumento è indirettamente indicato come una forma impropria di democrazia diretta, definita come "addomesticata". Nell'opera di Bruno Kaufmann invece lo strumento è energicamente rifiutato come strumento di democrazia diretta: "I plebisciti sono strumenti di potere nelle mani di chi governa, in cerca di approvazione da parte del Popolo per consolidare o salvare il proprio potere. Lo scopo non è tanto l’implementazione della democrazia, quanto piuttosto il dare legittimità alle decisioni di chi governa."
\end{itemize}


\section{Democrazia partecipativa}
La "democrazia partecipativa" consiste negli strumenti utili a raccogliere pareri e opinioni che forniscono informazioni stimolando la collaborazione tra cittadini e rappresentanti, ma di per sé questa forma di democrazia non contempla strumenti per attribuire potere legislativo ai cittadini.

\section{Lo strumento}
La caratteristica fondamentale di questo strumento è che non deve esserci l'amministratore e non deve necessariamnte esserci centralità dei dati. 
\chapter{Registrazione degli utenti}
\chapter{Struttura organizzativa}
\chapter{Votazione}
\chapter{Documenti condivisi e collaborazione}

\end{document}